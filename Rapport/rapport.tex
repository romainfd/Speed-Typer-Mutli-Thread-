\documentclass[a4paper,11pt]{article}
\usepackage[utf8]{inputenc}
\usepackage[french]{babel} 
\usepackage[T1]{fontenc} 
\usepackage{textcomp}
\usepackage{amsmath,amssymb}
\usepackage{mathrsfs}
\usepackage{stmaryrd}
\usepackage{graphicx}
\usepackage[titlepage,fancysections]{polytechnique}

\title{SpeedTyper}
\author{Vincent DALLARD et Romain FOUILLAND}
\subtitle{Projet d'INF431}
\date{Mars 2018}


\begin{document}
\maketitle
\section{Présentation}
\subsection{Introduction}
Le but du SpeedTyper est de taper le plus vite possible. Pour que le jeu soit pertinent, il faut donc que l'interface utilisateur (UI) soit la plus fluide possible. Ainsi, il faut gérer la vérification des mots et le décompte du temps en parallèle pour ne pas ralentir l'UI. De ce fait, un programme utilisant plusieurs threads est nécessaire.\par
Nous avons ainsi développée une première version naïve qui était gérée par l'UI. Afin d'améliorer les performances du jeu, nous avons dans une deuxième version utilisé des managers pour répartir les tâches et les nouveaux mots.\par

\subsection{Fonctionnement}
\end{document}
